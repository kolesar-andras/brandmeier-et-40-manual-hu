\documentclass{article}
\usepackage{lcd}
\usepackage{menukeys}
\usepackage[sfdefault]{plex-sans} % Use IBM Plex Sans as the sans font and make it the default
\usepackage{parskip}

\begin{document}

Hátulján levő kapcsolóval indítható el. Rövid ideig az alábbi szöveg látható:

\LCD{4}{20}
|  EINGABETASTATUR   |
|  ET 40-3  V3_22    |
|    BRANDMAIER      |
|   D-72160 HORB     |

Rövid idő után eltűnik, ekkor várja a megjelenítendő számjegyeket.

\textbf{Gombok jelentése}

\keys{0}-\keys{9} számjegyek

\keys{ } \emph{Enter}: a beírt számjegyeket ez küldi ki kijelzőre.

\keys{C} \emph{Clear = Lösch = törlés}: utoljára beírt számjegy és a teljes kijelző törlése

\keys{S} \emph{Stanza = Strophe = versszak}: ezzel lehet a kijelző második sorába írni

\keys{T} \emph{Text = Text = szöveg}: betűk írása

\keys{P} \emph{Programmierung = programozás}: előre be lehet állítani számok sorozatát

\keys{{+}}-\keys{{-}} írásjelek (+ és -) és belépés a beállítások menübe (együtt nyomva)

\textbf{Egyszerű példák:}

\keys{1} \keys{2} \keys{3} \keys{ } \hfill \LCD{2}{3}|123||   |

\keys{4} \keys{5} \keys{6} \keys{S} \keys{1} \keys{-} \keys{3} \keys{ } \hfill \LCD{2}{3}|456||1-3|

\end{document}
